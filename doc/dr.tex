\documentclass[11pt,a4paper]{article}

\usepackage{amssymb}
\usepackage{amsmath}
\usepackage{hyperref}
\hypersetup{breaklinks=true,colorlinks=true,linkcolor=blue,citecolor=blue,urlcolor=blue}
\usepackage{natbib}

\usepackage{fullpage}

\begin{document}

\author{S\'ebastien Villemot\thanks{Paris School of Economics and
    CEPREMAP. E-mail:
    \href{mailto:sebastien@dynare.org}{\texttt{sebastien@dynare.org}}.}}

\title{Solving rational expectations models at first order: \\
  what Dynare does\thanks{Copyright \copyright~2009,~2011 S\'ebastien
    Villemot. Permission is granted to copy, distribute and/or modify
    this document under the terms of the GNU Free Documentation
    License, Version 1.3 or any later version published by the Free
    Software Foundation; with no Invariant Sections, no Front-Cover
    Texts, and no Back-Cover Texts. A copy of the license can be found
    at: \url{https://www.gnu.org/licenses/fdl.txt}
    \newline
    The author acknowledges funding through the Seventh Framework Programme
    for Research (FP7) of the European Commission's Socio-economic Sciences and
    Humanities (SSH) Program under grant agreement SSH-CT-2009-225149.}
}

\date{First version: December 2009 \hspace{1cm} This version: April 2011}
\maketitle

\begin{abstract}
  This paper describes in detail the algorithm implemented in Dynare for
  computing the first order approximated solution of a nonlinear rational
  expectations model. The core of the algorithm is a generalized Schur
  decomposition (also known as the QZ decomposition), as advocated by several
  authors in the literature. The contribution of the present paper is to focus
  on implementation details that make the algorithm more generic and more
  efficient, especially for large models.

  \medskip
  \noindent
  \textit{JEL classification:} C63; C68

  \medskip
  \noindent
  \textit{Keywords:} Dynare; Numerical methods; Perturbation; Rational
  expectations

\end{abstract}

\section{Introduction}

Perturbation techniques are widely used for solving and estimating rational
expectation models and Dynare\footnote{Dynare is a free software platform for
  handling a wide class of economic models. See \url{http://www.dynare.org} and
  \citet{adjemian/al:2011} for more details.} offers a popular, user-friendly
access to these techniques. The purpose of the present paper is to describe in
detail the algorithm implemented in Dynare for computing the first order
approximated solution of nonlinear rational expectations models.\footnote{This
  algorithm is available using the \texttt{stoch\_simul} command of Dynare. The
  original implementation of this algorithm was done by Michel Juillard, using
  MATLAB, and is available in the \texttt{matlab/dyn\_first\_order\_solver.m} file which is
  distributed with Dynare.}

This algorithm is based on a generalized Schur decomposition---also known as
the QZ decomposition---and is therefore essentially a variation on the
methods presented by \citet{klein:2000}, \citet{sims:2001} and
\citet{uhlig:1999}.

The contribution of this paper is to present some implementation details that
make the algorithm more generic and more efficient for large models. In
particular I describe the algorithm for removing the leads and lags of more
than one in a nonlinear model. I also describe a way of reducing the size of
the Schur decomposition problem by dealing separately with endogenous variables
that appear only at the current date (called \emph{static} endogenous variables
in the following).

It should be noted that Dynare is able to go further than first order and can
deliver second and third order approximation of the solution of rational
expectations models. These higher order solutions can be computed recursively
using the first order solution as a starting point. For algorithmic details on
this issue, the interested reader can refer to
\citet{collard/juillard:2001:compecon} or \citet{schmitt-grohe/uribe:2004}.

The rest of this paper is organized as follows. Section \ref{sec:model}
presents the class of models to be solved and defines a typology of the
endogenous variables. Section \ref{sec:first-order} defines the solution
to the model and characterizes its first order approximation. Sections
\ref{sec:g-y} and \ref{sec:g-u} describe the algorithm used to recover this
first order approximation.

\section{The problem to be solved}
\label{sec:model}

\subsection{The model}

In the following, we consider a dynamic nonlinear rational expectations model
composed of several variables (endogenous and exogenous) and several
equations. In compact form, the model is written as:
\begin{equation}
  \label{eq:model}
  \mathbb{E}_t f(y^+_{t+1}, y_t, y^-_{t-1}, u_t) = 0
\end{equation}
where $y_t$ is the vector of endogenous variables, $y^+_{t+1}$
(resp. $y^-_{t-1}$) is the subset of variables of $y_t$ that appear with a lead
(resp. a lag), and $u_t$ is the vector of exogenous variables. For obvious
identification reasons, the model must contain as many equations as there are
endogenous variables; let $n$ be this number.

For the timing of variables, the convention used here is the same as in Dynare:
a variable decided at date $t$ should have a time subscript of $t$. For
example, in a typical RBC model, the capital stock used to produce date $t$
output is actually decided at date $t-1$, so it will be written as $k_{t-1}$
using this convention. Thus accordingly, the law of motion of capital will be
$k_t = (1-\delta)k_{t-1} + i_t$. Another way of expressing this timing
convention is that stock variables should use the ``end-of-period'' convention.

The vector of exogenous variables $u_t$ (of dimension $p$) follows a Markov
process:
\begin{equation*}
  u_t = P(u_{t-1}, \varepsilon_t)
\end{equation*}
where the $\varepsilon_t$ are independent and identically distributed
innovations with zero mean and variance-covariance matrix $\Sigma$.

Note that the stochastic process adopted here for exogenous variables is more
generic than the one allowed in Dynare (which only accepts a white noise for
stochastic variables, \textit{i.e.} $u_t = \varepsilon_t$).

\subsection{Typology of variables}

All endogenous variables are required to appear at least at one period. However
it is not required that all endogenous variables appear at the current period
(a weaker condition is actually sufficient, see assumption
(\ref{eq:identification}) below).

We define four types of endogenous variables:
\begin{description}
\item[Static endogenous variables:] those that appear only at the current
  period. Their number is $n^s \leq n$, and their indices in the $y_t$ vector
  are $\zeta^s_j, j=1\ldots n^s$
\item[Purely forward endogenous variables:] those that appear only at the
  future period, possibly at the current period, but not at the previous
  period. Their number is $n^{++} \leq n$, and their indices $\zeta^{++}_j,
  j=1\ldots n^{++}$
\item[Purely backward endogenous variables:] those that appear only at the
  previous period, possibly at the current period, but not at the future
  period.  Their number is $n^{--} \leq n$, and their indices $\zeta^{--}_j,
  j=1\ldots n^{--}$
\item[Mixed endogenous variables:] those that appear both at the future and the
  previous period, and possibly at the current period.  Their number is $n^m
  \leq n$, and their indices $\zeta^m_j, j=1\ldots n^m$
\end{description}

These four types of variables form a partition of the endogenous variables, and
we therefore have:
\begin{equation*}
  n^m + n^{++} + n^{--} + n^s = n
\end{equation*}

We also define:
\begin{description}
\item[Forward endogenous variables:] the union of mixed and purely forward
  endogenous variables. Their number is $n^+ = n^{++} + n^m$, and their indices
  $\zeta^+_j, j=1\ldots n^+$.
\item[Backward endogenous variables:] the union of mixed and purely backward
  endogenous variables. Their number is $n^- = n^{--} + n^m$, and their indices
  $\zeta^-_j, j=1\ldots n^-$
\item[Dynamic endogenous variables:] all the variables except static endogenous
  variables. Their number is $n^d = n - n^s$, and their indices $\zeta^d_j,
  j=1\ldots n^d$
\end{description}

The seven indices are such that $1 \leq \zeta^k_1 < \zeta^k_2 < \ldots <
\zeta^k_{n^k} \leq n$, where $k \in \{ s, +, ++, -, --, m, d \}$.

We denote by $y_t = (y_{1,t}, \ldots, y_{n,t})'$ the vector of endogenous
variables at date $t$. We denote by $y^k_t = (y_{\zeta^k_1, t}, \ldots,
y_{\zeta^k_{n^k}, t})'$ a subvector of endogenous variables, where $k \in \{
s, +, ++, -, --, m, d \}$.

We denote by $\beta^+_j, j=1\ldots n^m$ the indices of mixed endogenous
variables inside the $\zeta^+_j$ sequence, \textit{i.e.} $\beta^+_j$ is such
that $\zeta^+_{\beta^+_j}$ is a mixed endogenous variable. We similarly define
$\beta^-_j$ for mixed endogenous variables inside the $\zeta^-_j$ sequence. We
similarly define $\pi^+_j$ (resp. $\pi^-_j$) for purely forward (resp. purely
backward) endogenous variables inside $\zeta^+_j$ (resp $\zeta^-_j$).

Finally, the vector of \emph{state variables} is formed by the union of
backward endogenous variables at the previous period and of exogenous variables
at the current period, and is therefore of size $n^-+p$.

\subsection{Removing extra leads and lags}

The form given in equation (\ref{eq:model}) makes the assumption that
endogenous variables appear with at most one lead and one lag, and
that exogenous variables appear only at the current period. This
assumption does not imply any loss of generality, since it is easy to
transform a nonlinear model with many leads and lags into an
equivalent model of the form given in (\ref{eq:model}), as is detailed
below.\footnote{The algorithm described in the present section is
  implemented in the Dynare preprocessor, since version 4.1. Auxiliary
  variables are created automatically and will show up at several
  places in Dynare output; see \citet{adjemian/al:2011} for the names
  of these variables.}

For every variable $x_t$ in the original model whose maximum lag is $x_{t-d-k}$
with $k>0$ (and $d=1$ if $x$ is an endogenous variable or $d=0$ if it is an
exogenous variable), the transformation is the following:
\begin{itemize}
\item introduce $k$ new endogenous variables $z^j_t$, for
  $j\in\{1,\ldots,k\}$;
\item add $k$ new equations to the model:
  \begin{equation*}
    \left\{\begin{array}{ll}
        z^1_t = x_{t-d} & \\
        z^j_t = z^{j-1}_{t-1} & \text{for } j\in\{2, \ldots,k\}
      \end{array}\right.;
  \end{equation*}
\item replace all occurrences of $x_{t-d-j}$ (with $j>0$) in the original model
  by $z^j_{t-1}$ in the transformed model.
\end{itemize}

The transformation for variables with a lead is a bit more elaborate because
one has to handle the fact that there is an expectation operator in front of
all equations. The algorithm is as follows:
\begin{itemize}
\item decompose every equation of the original model in the following form:
  \begin{equation*}
    A + \sum_{i\in I} B_i \, \mathbb{E}_t C_i = 0
  \end{equation*}
  where $A$ and the $B_i$ are (possibly nonlinear) expressions containing only
  current or lagged variables, and the $C_i$ are (possibly nonlinear)
  expressions which may contain leads; this decomposition is not unique, but
  one should aim at making the $C_i$ terms as simple as possible;
\item for every $C_i$ where there is a lead of 2 or more on an endogenous
  variable, or a lead on an exogenous variable:
  \begin{itemize}
  \item let $k$ be the minimal number of periods so that $C_i^{(-k)}$ has at
    most one lead on endogenous variables and no lead on exogenous variables
    (where $C_i^{(-k)}$ stands for the transformation of $C_i$ where all
    variables have been lagged by $k$ periods);
  \item introduce $k$ new endogenous variables $z^j_t$, for
    $j\in\{1,\ldots,k\}$;
  \item add $k$ new equations to the model:
    \begin{equation*}
      \left\{\begin{array}{ll}
          z^1_t = \mathbb{E}_tC_i^{(-k)} & \\
          z^j_t = \mathbb{E}_tz^{j-1}_{t+1} & \text{for } j\in\{2, \ldots,k\}
        \end{array}\right.;
    \end{equation*}
  \item replace all occurrences of $\mathbb{E}_t C_i$ in the original model
    by $\mathbb{E}_tz^k_{t+1}$ in the transformed model.
  \end{itemize}
\end{itemize}
It is straightforward to see that this transformed model is under the form
given in (\ref{eq:model}). And by the law of iterated expectations, it is
equivalent to the original one.

\section{The solution and its first order approximation}
\label{sec:first-order}

We first define the deterministic steady state of the model as the vector
$(\bar{y}, \bar{u}, \bar{\varepsilon})$ satisfying:
\begin{equation*}
  \bar{\varepsilon} = 0
\end{equation*}
\begin{equation*}
  \bar{u} = P(\bar{u}, \bar{\varepsilon})
\end{equation*}
\begin{equation*}
  f(\bar{y}^+, \bar{y}, \bar{y}^-, \bar{u}) = 0
\end{equation*}
Finding the deterministic steady state involves the resolution of a
multivariate nonlinear system.\footnote{Dynare offers efficient ways of
  performing this task, but this is out of the scope of the present paper.}
Then, finding the rational expectation solution of the model means finding the
policy functions (also known as decision rules), which give current endogenous
variables as a function of state variables:
\begin{equation*}
  y_t = g(y^-_{t-1}, u_t)
\end{equation*}
Note that, by definition of the deterministic steady state, we have $\bar{y} =
g(\bar{y}^-, \bar{u})$.

The function $g$ is characterized by the following functional equation:
\begin{equation}
  \label{eq:g-definition}
  \mathbb{E}_t f\left[g^+(g^-(y^-_{t-1}, u_t), u_{t+1}), g(y^-_{t-1}, u_t), y^-_{t-1}, u_t\right] = 0
\end{equation}
where $g^+$ (resp. $g^-$) is the restriction of $g$ to forward (resp. backward)
endogenous variables.

In the general case, this functional equation cannot be solved exactly, and one
has to resort to numerical techniques to get an approximated solution. The
remainder of this paper describes the first order perturbation technique
implemented in Dynare. Let:
\begin{equation*}
  f_{y^+} = \frac{\partial f}{\partial y^+_{t+1}}, \; f_{y^0} = \frac{\partial f}{\partial y_t}, \;  f_{y^-} = \frac{\partial f}{\partial y^-_{t-1}}, \;
  f_u   = \frac{\partial f}{\partial u_t}
\end{equation*}
\begin{equation*}
  g_y = \frac{\partial g}{\partial y^-_{t-1}},\;
  g_u = \frac{\partial g}{\partial u_t}
\end{equation*}
\begin{equation*}
  P_u = \frac{\partial P}{\partial u_{t-1}}, \:
  P_\varepsilon = \frac{\partial P}{\partial \varepsilon_t}
\end{equation*}
where the derivatives are taken at $\bar{y}$, $\bar{u}$ and
$\bar{\varepsilon}$.

The first order approximation of the policy function is therefore:
\begin{equation*}
  \hat{g}(y^-_{t-1}, u_t) = \bar{y} + g_y \hat{y}^-_{t-1} + g_u \hat{u}_t
\end{equation*}
where $\hat{y}^-_{t-1} = y^-_{t-1} - \bar{y}^-$, $\hat{u}_t = u_t - \bar{u}$,
and $g_y$ and $g_u$ are unknowns at this stage.

A first order approximation of (\ref{eq:g-definition}) around $\bar{y}$ and
$\bar{u}$ gives:
\begin{multline*}
  f(\bar{y}^+, \bar{y}, \bar{y}^-, \bar{u}) + f_{y^+} [g^+_y(g^-_y
  \hat{y}^-_{t-1} + g^-_u \hat{u}_t) + g^+_u \mathbb{E}_t [P_u \hat{u}_t +
  P_\varepsilon \varepsilon_{t+1}] ] \\ + f_{y^0} (g_y \hat{y}^-_{t-1} + g_u \hat{u}_t) + f_{y^-}\hat{y}^-_{t-1} + f_u \hat{u}_t = 0
\end{multline*}
where $g^+_y$, $g^-_y$, $g^-_u$, $g^+_u$ are the derivatives of the
restrictions of $g$ with obvious notation. Computing the expectancy term,
taking into account the property of the deterministic steady state, and
reorganizing the terms, we obtain:
\begin{equation}
  \label{eq:first-order}
  (f_{y^+} g^+_y g^-_y + f_{y^0}  g_y + f_{y^-}) \hat{y}^-_{t-1} + (f_{y^+} g^+_yg^-_u+ f_{y^+}g^+_u P_u + f_y g_u + f_u) \hat{u}_t = 0
\end{equation}

In the next sections, we exploit this equation in order to recover the unknown
coefficients $g_u$ and $g_y$.

\section{Recovering $g_y$}
\label{sec:g-y}

Taking into account the term multiplying $\hat{y}^-_{t-1}$, equation
(\ref{eq:first-order}) imposes:
\begin{equation*}
  f_{y^+} g^+_y g^-_y + f_{y^0}  g_y + f_{y^-} = 0
\end{equation*}
This amounts to:
\begin{equation}
  \label{eq:gy}
  f_{y^+} \hat{y}^+_{t+1} + f_{y^0} \hat{y}_t + f_{y^-} \hat{y}^-_{t-1} = 0
\end{equation}

Let $S$ be the $n\times n^s$ submatrix of $f_{y^0}$ where only the columns for
static endogenous variables are kept, \textit{i.e.} $S_{i,j} = f_{y^0, i,
  \zeta^s_j}$. A QR decomposition\footnote{See \citet[section
  5.2]{golub/van-loan:1996}.} gives $S = QR$ where $Q$ is an $n\times n$
orthogonal matrix and $R$ an $n\times n^s$ upper triangular matrix.

For the model to be identified, we assume that:
\begin{equation}
  \label{eq:identification}
  \mathop{rank}(R) = n^s.
\end{equation}
Thus, equation (\ref{eq:gy}) can be rewritten as:
\begin{equation}
  \label{eq:gy-qr}
  A^+ \hat{y}^+_{t+1} + A^0 \hat{y}_t + A^- \hat{y}^-_{t-1} = 0
\end{equation}
where $A^+ = Q'f_{y^+}$, $A^0 = Q'f_{y^0}$ and $A^- = Q'f_{y^-}$. By
construction, columns of static endogenous variables in $A^0$ are zero in their
lower part: $\forall i > n^s,\forall j\leq n^s, \: A^0_{i,\zeta^s_j} = 0$.

\subsection{Non-static endogenous variables}

Taking only the $n^d$ lower rows of system (\ref{eq:gy-qr}), we get:
\begin{equation}
  \label{eq:gy-no-static}
  \tilde{A}^+ \hat{y}^+_{t+1} + \tilde{A}^{0+} \hat{y}^+_t + \tilde{A}^{0-} \hat{y}^-_t + \tilde{A}^- \hat{y}^-_{t-1} = 0
\end{equation}
where $\tilde{A}^+$ (resp. $\tilde{A}^-$) contains the last $n^d$ rows of $A^+$
(resp. $A^-$). Matrices $\tilde{A}^{0+}$ and $\tilde{A}^{0-}$ can be defined in
two ways, depending on where we deal with mixed endogenous variables:

\begin{itemize}
\item $\tilde{A}^{0+}$ is a submatrix of $A^0$ where only the last $n^d$ rows
  and the columns for forward endogenous variables are kept
  ($\tilde{A}^{0+}_{i,j} = A^0_{n^s+i, \zeta^+_j}$), and $\tilde{A}^{0-}$ is
  such that purely backward columns are taken from $A^0$
  ($\tilde{A}^{0-}_{i,\pi^-_j} = A^0_{n^s+i,\zeta^{--}_j}$), and the rest is
  zero;
\item $\tilde{A}^{0-}$ is a submatrix of $A^0$ where only the last $n^d$ rows
  and the columns for backward endogenous variables are kept
  ($\tilde{A}^{0-}_{i,j} = A^0_{n^s+i, \zeta^-_j}$), and $\tilde{A}^{0+}$ is
  such that purely forward columns are taken from $A^0$
  ($\tilde{A}^{0+}_{i,\pi^+_j} = A^0_{n^s+i,\zeta^{++}_j}$), and the rest is
  zero.
\end{itemize}
Note that in equation (\ref{eq:gy-no-static}), static endogenous variables no
longer appear.

The structural state space representation of (\ref{eq:gy-no-static}) is:

\begin{equation*}
  \underbrace{
    \left(
      \begin{matrix}
        \tilde{A}^{0-} & \tilde{A}^+ \\
        I^- & 0
      \end{matrix}
    \right)
  }_D
  \left(
    \begin{matrix}
      \hat{y}^-_t \\
      \hat{y}^+_{t+1}
    \end{matrix}
  \right)
  =
  \underbrace{
    \left(
      \begin{matrix}
        -\tilde{A}^- & -\tilde{A}^{0+} \\
        0 & I^+
      \end{matrix}
    \right)
  }_E
  \left(
    \begin{matrix}
      \hat{y}^-_{t-1} \\
      \hat{y}^+_t
    \end{matrix}
  \right)
\end{equation*}
where $I^-$ is an $n^m \times n^-$ selection matrix for mixed endogenous
variables, such that $I^-_{i,\beta^-_i}=1$, and zero otherwise. Similarly,
$I^+$ is an $n^m \times n^+$ matrix, such that $I^+_{i,\beta^+_i}=1$, and zero
otherwise. Therefore, $D$ and $E$ are square matrices of size
$n^{++}+n^{--}+2n^m$.

Using the fact that $\hat{y}^+_{t+1} = g^+_y \hat{y}^-_t$, this can be
rewritten as:
\begin{equation}
  \label{eq:state-space}
  D
  \left(
    \begin{matrix}
      I_{n^-} \\
      g^+_y
    \end{matrix}
  \right)
  \hat{y}^-_t
  =
  E
  \left(
    \begin{matrix}
      I_{n^-} \\
      g^+_y
    \end{matrix}
  \right)
  \hat{y}^-_{t-1}
\end{equation}
where $I_{n^-}$ is the identity matrix of size $n^-$.

A generalized Schur decomposition (also known as the QZ decomposition) of the
pencil $(D,E)$ is performed:\footnote{See \citet[section
  7.7]{golub/van-loan:1996} for theoretical and practical details on this
  decomposition.}
\begin{equation*}
  \left\{\begin{array}{rcl}
      D & = & QTZ \\
      E & = & QSZ
    \end{array}
  \right.
\end{equation*}
where $T$ is upper triangular, $S$ quasi upper triangular, and $Q$ and $Z$ are
orthogonal matrices. The decomposition is done is such a way that stable
generalized eigenvalues (modulus less than 1) are in the upper left corner of
$T$ and $S$.

Matrices $T$ and $S$ are block decomposed so that the upper left block of both
matrices is square and contains generalized eigenvalues of modulus less than 1,
and the lower right block is square and contains generalized eigenvalues of
modulus strictly greater than 1.

Equation (\ref{eq:state-space}) can be rewritten as:
\begin{equation}
  \label{eq:state-space-qz}
  \left(
    \begin{matrix}
      T_{11} & T_{12} \\
      0 & T_{22}
    \end{matrix}
  \right)
  \left(
    \begin{matrix}
      Z_{11} & Z_{12} \\
      Z_{21} & Z_{22}
    \end{matrix}
  \right)
  \left(
    \begin{matrix}
      I_{n^-} \\
      g^+_y
    \end{matrix}
  \right)
  \hat{y}^-_t
  =
  \left(
    \begin{matrix}
      S_{11} & S_{12} \\
      0 & S_{22}
    \end{matrix}
  \right)
  \left(
    \begin{matrix}
      Z_{11} & Z_{12} \\
      Z_{21} & Z_{22}
    \end{matrix}
  \right)
  \left(
    \begin{matrix}
      I_{n^-} \\
      g^+_y
    \end{matrix}
  \right)
  \hat{y}^-_{t-1}
\end{equation}
where $T_{11}$ and $S_{11}$ are square and contain stable generalized
eigenvalues, while $T_{22}$ and $S_{22}$ are square and contain explosive
generalized eigenvalues.

To exclude explosive trajectories, we impose:
\begin{equation}
  \label{eq:non-explosive}
  \left(
    \begin{matrix}
      Z_{11} & Z_{12} \\
      Z_{21} & Z_{22}
    \end{matrix}
  \right)
  \left(
    \begin{matrix}
      I_{n^-} \\
      g^+_y
    \end{matrix}
  \right)
  =
  \left(
    \begin{matrix}
      X \\
      0
    \end{matrix}
  \right)
\end{equation}
which implies:
\begin{equation*}
  g^+_y = -(Z_{22})^{-1} Z_{21}
\end{equation*}

Note that the squareness of $Z_{22}$ is the \citet{blanchard/kahn:1980}
\emph{order} condition (\textit{i.e.} the requirement to have as many explosive
eigenvalues as forward or mixed endogenous variables), and the non-singularity
of $Z_{22}$ is the \citet{blanchard/kahn:1980} \emph{rank} condition.

Using equation (\ref{eq:non-explosive}) and the fact that $\hat{y}^-_t = g^-_y
\hat{y}^-_{t-1}$, equation (\ref{eq:state-space-qz}) implies:
\begin{equation*}
  \left(
    \begin{matrix}
      T_{11} & T_{12} \\
      0 & T_{22}
    \end{matrix}
  \right)
  \left(
    \begin{matrix}
      X \\
      0
    \end{matrix}
  \right)
  g^-_y
  =
  \left(
    \begin{matrix}
      S_{11} & S_{12} \\
      0 & S_{22}
    \end{matrix}
  \right)
  \left(
    \begin{matrix}
      X \\
      0
    \end{matrix}
  \right)
\end{equation*}
Then, using the fact that solving equation (\ref{eq:non-explosive}) for $X$
gives $X = (Z'_{11})^{-1}$, the upper part of this system gives the solution
for $g^-_y$:
\begin{equation*}
  g^-_y = X^{-1} T_{11}^{-1}S_{11}X = Z'_{11}T_{11}^{-1}S_{11}(Z'_{11})^{-1}
\end{equation*}

Note that mixed variables appear in both $g^+$ and $g^-$: the corresponding
lines will be equal across the two matrices by construction.

\subsection{Static endogenous variables}

The $n^s$ upper lines of equation (\ref{eq:gy-qr}) can be written as:
\begin{equation}
  \label{eq:static-part}
  \breve{A}^+ \hat{y}^+_{t+1} + \breve{A}^{0d} \hat{y}^d_t + \breve{A}^{0s} \hat{y}^{s}_t + \breve{A}^- \hat{y}^-_{t-1} = 0
\end{equation}
where $\breve{A}^+$ (resp. $\breve{A}^-$) contains the first $n^s$ rows of
$A^+$ (resp. $A^-$). Matrix $\breve{A}^{0s}$ (resp. $\breve{A}^{0d}$) contains
the first $n^s$ rows and only the static (resp. non-static) columns of
$A^0$. Recall that $\breve{A}^{0s}$ is a square upper triangular matrix by
construction, and it is invertible because of assumption
(\ref{eq:identification}).

Equation (\ref{eq:static-part}) can be rewritten as:
\begin{equation*}
  \breve{A}^+ g^+_y g^-_y \hat{y}^-_{t-1} + \breve{A}^{0d} g^d_y \hat{y}^-_{t-1} + \breve{A}^{0s} \hat{y}^{s}_t + \breve{A}^- \hat{y}^-_{t-1} = 0
\end{equation*}
where $g^d_y$, the restriction of $g_y$ to non-static endogenous variables, is
obtained by combining $g^+_y$ and $g^-_y$. We therefore have:
\begin{equation*}
  g^s_y = -\left[\breve{A}^{0s}\right]^{-1} \left(\breve{A}^+ g^+_y g^-_y + \breve{A}^{0d} g^d_y + \breve{A}^-\right)
\end{equation*}

\section{Recovering $g_u$}
\label{sec:g-u}

Equation (\ref{eq:first-order}) restricted to $\hat{u}_t$ imposes:
\begin{equation*}
  f_{y^+} g^+_yg^-_u+ f_{y^+}g^+_u P_u + f_y g_u + f_u = 0,
\end{equation*}
and be rewritten as:
\begin{equation*}
  (f_{y^+} g^+_y J^- + f_y) g_u + f_{y^+}J^+ g_u P_u + f_u = 0
\end{equation*}
where $J^-$ (resp $J^+$) is an $n^-\times n$ matrix (resp. $n^+\times n$
matrix) selecting only the backward (resp. forward) endogenous variables.  In
the particular case solved by Dynare, where $P_u = 0$, the solution to this
equation is:
\begin{equation*}
  g_u = -(f_{y^+} g^+_y J^- + f_y)^{-1} f_u
\end{equation*}
In the general case, this equation is a specialized Sylvester equation, which
can be solved using the algorithm proposed by
\citet{kamenik:2004}\footnote{This paper is distributed with Dynare, in the
  \texttt{sylvester.pdf} file under the Dynare++ documentation directory.}.

\bibliographystyle{elsarticle-harv}
\bibliography{dr}

\end{document}
